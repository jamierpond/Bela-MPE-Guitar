
\tikzset{->-/.style={decoration={
  markings,
  mark=at position #1 with {\arrow{>}}},postaction={decorate}}}
  
\begin{figure}[h]
    \centering
    
    \begin{tikzpicture}[scale=2.2]
    \clip(-0.60,-0.60) rectangle (3.40,3.45);
    \tkzInit[xmax=4,ymax=4,xmin=0,ymin=0]
    \tkzGrid
    \tkzAxeXY
    \node at (1.75, -0.5){\large $k_1$};
    \node[] at (-0.5, 1.75){\large $k_2$};
    
    % Draws the first blue line and red circle.
    \draw[thick,->-=.5,blue] (0,0) -- (1,0);
    \filldraw (0,0)[red] circle (1.7pt);
    
    % <THIS DRAWS THE BLUE LINES>
    \newcounter{N}{0}
    \foreach \n in {1,...,15} {
        \foreach \i in {0,...,\value{N}}{
            \draw[thick, ->-=.5,blue] (\n - \i, \i) -- (\n - 1 - \i ,\i + 1);
        }
        \stepcounter{N}
        \draw[thick, ->-=.5,blue] (0,\n) -- (\n+1,0); %IMPORTANT
    } % END <THIS DRAWS THE BLUE LINES>
    
    % <THIS DRAWS THE NUMBER AND RED DOTS>
    % These are 'registers' used to calculate the cantor number. 
    \setcounter{N}{3}
    \newcounter{cn}{0}
    \newcounter{r1}{0}
    \newcounter{r2}{0}
    \newcounter{r3}{0}
    \foreach \n in {0,...,3} {
        \foreach \i in {0,...,\value{N}}{
            % Fill circles. 
            \filldraw (\i, \n)[red] circle (1.5pt);
            
            %% CALCULATE CANTOR NUMBER
            % Reset registers. 
            \setcounter{cn}{0}
            \setcounter{r1}{\i}
            \setcounter{r2}{\n}
            \setcounter{r3}{0}
            
            % cn = (k1 + k2 + 1)
            \addtocounter{cn}{\value{r1}}
            \addtocounter{cn}{\value{r2}}
            \addtocounter{cn}{1}
            
            % r3 = (k1 + k2)
            \addtocounter{r3}{\value{r1}}
            \addtocounter{r3}{\value{r2}}
            
            % cn *= r3
            \multiply\value{cn} by \value{r3}
            
            % cn /= 2
            \divide\value{cn} by 2
            
            % cn += k2
            \addtocounter{cn}{\value{r2}}
            %% END CALCULATE CANTOR NUMBER
            
            % Display cantor number. 
            \node at (\i + 0.125, \n + 0.125){\arabic{cn}};
        }
    } % END <THIS DRAWS THE NUMBERS AND RED DOTS>
   
\end{tikzpicture}
    \caption{Visualization of the two element Cantor pairing algorithm.}
    \label{fig:my_label}
\end{figure}