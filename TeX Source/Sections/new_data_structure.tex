\begin{figure}[h]
\centering
\begin{tikzpicture}[every node/.style={anchor=north east, fill=white, minimum width=0.5cm, minimum height=5mm}]



% OFFSETS
\matrix (mA) [draw,matrix of math nodes]
{
0.00 & 0.00 & 0.00 & \textbf{0.63} & 0.00 & 0.00 \\
\textbf{0.08} & 0.00 & 0.00 & 0.00 & 0.00 & 0.00 \\
0.00 & 0.00 & 0.00 & 0.00 & 0.00 & \textbf{0.91} \\
};

% ONSETS
\matrix (mB) [draw,matrix of math nodes] at ($(mA.south west)+(2.9,0.7)$)
{
0.00 & 0.00 & 0.00 & \textbf{0.50} & 0.00 & 0.00 \\
\textbf{0.50} & 0.00 & 0.00 & 0.00 & 0.00 & 0.00 \\
0.00 & 0.00 & 0.00 & 0.00 & 0.00 & \textbf{0.50} \\
};

% C
\matrix (mC) [draw,matrix of math nodes] at ($(mB.south west)+(2.5,0.7)$)
{
0 & 0 & 0 & \textbf{1} & 0 & 0 \\
\textbf{1} & 0 & 0 & 0 & 0 & 0 \\
0 & 0 & 0 & 0 & 0 & \textbf{1} \\
};

\node[align=left] at ($(mC.south east) + (-0.4cm, -0cm)$) {\texttt{isActive}};
\node[align=left] at ($(mB.south east) + (-0.5cm, -0cm)$) {\texttt{touchSize}};
\node[align=left] at ($(mA.south east) + (0.6cm, -0cm)$) {\texttt{touchLocation}};
\end{tikzpicture}

\caption{A visual representation of the updated data structure used to represent touch information. }
\label{fig:new_struct}

\end{figure}