\begin{abstract}

We explore the development of a prototype MIDI Polyphonic Expression (MPE) enabled, guitar-like instrument for use in the modern music production studio. Following a process of human-centred design and human-computer interaction, we focus on creating a prototype that is conducive to fostering creative flow. This is achieved through the implementation of capacitive touch sensors (Bela Trill Bar) which are affixed to the frets of the prototype, which offers affordances of polyphonic aftertouch and pitch bend. The data thereof is interpreted with a given mathematical and programming framework. We find that the prototype is proficient at supporting novel creative musical ideas and subjectively pleasingly idiosyncratic musical performances, however this is somewhat impeded by practical prototyping choices. 

A brief demonstration of the working prototype is  \href{https://drive.google.com/file/d/1nzG5n9ZV4Jj0LsfHVGwRh3t7zOLOUAXv/view?usp=sharing}{\textit{available here}}.


% In this paper we will explore the development of a prototype MIDI Polyphonic Expression (MPE) enabled digital instrument. This work focuses on the process of human-centred design and human-computer interaction with a focus on creating instruments that are not only conducive to fostering flow in musicians, but also ensuring an accessible design for blind, limb-different and left-handed users. This work also attempts to tackle the stigma of digital instruments that has been highlighted in the literature, expressing how some musicians in general avoid using digital interfaces since they find their design is not conducive to their creative flow. 


\end{abstract}

\begin{IEEEkeywords}
Augmented guitar, digital instruments, MIDI, NIME, HCI
\end{IEEEkeywords}



% In this paper, we build upon the guitar as an \textit{instrument-like controller}, offering a new palate of gestures for MPE to extend the expressive vocabulary for the guitar as a MIDI instrument. We explore the practical design and prototyping of such an instrument, addressing the themes of ambidextrous design and accessibility for limb-different users.